\chapter{Introdução}
\label{c.introducao}

Com o grande crescimento que se deu na tecnologia nos últimos anos, surgiram novas maneiras para sanar quesitos de gastos e desempenho quanto às necessidades básicas. Os sistemas em nuvem ({\em Cloud Services}) surgiram de modo a terceirizar o hardware utilizado, oferecendo recursos de hardware que podem ser alugados, sendo que o cliente paga apenas os recurso que são utilizados, deixando para este apenas se preocupar com o produto que será comercializado.

Como esses serviços se encarregam de manter a hospedagem de dados, além de manter o tráfico de usuários não congestionado, o cliente não se preocupa também com as ameaças que possa vir a ter. Nesse ponto, o sistema na nuvem terá que oferecer um sistema invunerável, para manter credibilidade.

Testes de Penetração, também conhecidos como {\em PenTest} ou Testes de Invasão, consistem em recolher informações sobre o alvo, identificar possíveis aberturas, tentativas de invasão e relatórios sobre o teste propriamente dito. O objetivo principal de um pentest é de determinar pontos fracos na segurança do sistema.

\section{Problema}
\label{s.problema}

A cada dia que passa, novas tecnologias são criadas ou mesmo novos métodos são descobertos. Porém isso pode ser beneficiente para a sociedade como também para pessoas mal intencionadas. Tendo isso em vista, é necessário manter em dia conhecimento de todas as possíveis vulnerabilidades que um sistema possa a ter, para que não haja roubo de informações e dados sigilosos. Ao oferecer um serviço na nuvem, a empresa precisa garantir à seus clientes que não haverá invasão de terceiros, não havendo preocupação dos mesmos com os dados que eles disponibilizam para os serviços na nuvem.

% \section{Abordagem Colaborativa}
% \label{s.abordagem}
%
% Segundo \citeauthoronline{unite10} (\citeyear{unite10}, Tradução nossa), projetos colaborativos permitem a criação conjunta e simultânea de conteúdo por muitos usuários finais e são, nesse sentido, provavelmente a mais democrática manifestação de conteúdo gerado pelo usuário.
%
% Ao desenvolver uma ferramenta que seja modular, é possível garantir uma maior coesão entre os componentes dessa ferramenta, pois cada módulo tem uma função definida e uma definição de formato de entrada e saída de dados.
%
% Por se trabalhar com código aberto o conteúdo disponível é altamente modificável, pois o código faz parte da implementação e da transmissão do conhecimento.
% Essa possibilidade de modificação faz com que o código seja continuamente refatorado e revisto, e dentro de padrões de desenvolvimento tende a melhorar sua qualidade final.
%
% Como diz \citeauthoronline{weber} (\citeyear{weber}, Tradução nossa) o código aberto não oblitera o lucro, o capitalismo, ou direitos intelectuais, e portanto não vai contra os modelos atuais de mercado, inclusive auxilia nos processos de todas as áreas sociais. O próprio formato do HTML\footnote{HyperText Markup Language (Linguagem de Marcação de Hipertexto).} e essa ampla distribuição de código aberto através da internet tem como efeito uma grande divulgação de linguagens e ferramentas open source. Ferramentas que antes precisavam de um conhecimento mais amplo em computação, agora podem ser manipuladas com uma chamada do script direto no navegador de quem está testando um pedaço de código.
%
% A utilização do formato modularizado garante também uma liberdade e uma intercomunicação entre trabalhos já desenvolvidos, como cálculos de transformadas e algoritmos de aprendizagem.%colocar mais libs usadas aqui
