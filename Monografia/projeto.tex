\documentclass[
	% -- opções da classe memoir --
	12pt,				% tamanho da fonte
	openright,			% capítulos começam em pág ímpar (insere página vazia caso preciso)
	oneside,			% para impressão em verso e anverso. Oposto a oneside
	a4paper,			% tamanho do papel.
	% -- opções da classe abntex2 --
	%chapter=TITLE,		% títulos de capítulos convertidos em letras maiúsculas
	%section=TITLE,		% títulos de seções convertidos em letras maiúsculas
	%subsection=TITLE,	% títulos de subseções convertidos em letras maiúsculas
	%subsubsection=TITLE,% títulos de subsubseções convertidos em letras maiúsculas
	% -- opções do pacote babel --
	english,			% idioma adicional para hifenização
	french,				% idioma adicional para hifenização
	spanish,			% idioma adicional para hifenização
	brazil				% o último idioma é o principal do documento
	]{abntex2}

% ---
% Pacotes básicos
% ---
\usepackage{lmodern}			% Usa a fonte Latin Modern
\usepackage[T1]{fontenc}		    % Selecao de codigos de fonte.
\usepackage[utf8]{inputenc}		% Codificacao do documento (conversão automática dos acentos)
\usepackage{lastpage}			% Usado pela Ficha catalográfica
\usepackage{indentfirst}		    % Indenta o primeiro parágrafo de cada seção.
\usepackage{color}				% Controle das cores
\usepackage{graphicx}			% Inclusão de gráficos
\usepackage{microtype} 			% Para melhorias de justificação
\usepackage{afterpage}
\usepackage{amsmath}            % Pacote para fórmulas matemáticas
\usepackage{amssymb,url}
\usepackage{xcolor,tikz,bm,colortbl}
\usepackage[br]{nicealgo}       % Pacote para criação de algoritmos
\usepackage{customizacoes}

% ---

% ---
% Pacotes adicionais, usados apenas no âmbito do Modelo Canônico do abnteX2
% ---
\usepackage{lipsum}				% Para geração de dummy text
% ---

% ---
% Pacotes de citações
% ---
\usepackage[brazilian,hyperpageref]{backref}	 % Paginas com as citações na bibl
\usepackage[alf]{abntex2cite}	% Citações padrão ABNT
% ---
% CONFIGURAÇÕES DE PACOTES
% ---

% ---
% Configurações do pacote backref
\renewcommand{\familydefault}{\sfdefault}
% Usado sem a opção hyperpageref de backref
\renewcommand{\backrefpagesname}{Citado na(s) página(s):~}
% Texto padrão antes do número das páginas
\renewcommand{\backref}{}
% Define os textos da citação
\renewcommand*{\backrefalt}[4]{
	\ifcase #1 %
		Nenhuma citação no texto.%
	\or
		Citado na página #2.%
	\else
		Citado #1 vezes nas páginas #2.%
	\fi}%
% ---

% ---
% Informações de dados para CAPA e FOLHA DE ROSTO
% ---
\titulo{Métodos de Teste de Penetração para Sistemas Web na Nuvem}
\autor{Hugo Cicarelli}
\local{Bauru}
\data{2016}
\orientador{Prof. Dr. Kelton Augusto Pontara da Costa}
\instituicao{%
  Universidade Estadual Paulista ``Júlio de Mesquita Filho''
  \par
  Faculdade de Ciências
  \par
  Ciência da Computação}
\tipotrabalho{Trabalho de Conclusão de Curso}
% O preambulo deve conter o tipo do trabalho, o objetivo,
% o nome da instituição e a área de concentração
\preambulo{Trabalho de Conclusão de Curso do Curso de Ciência da Computação da Universidade Estadual Paulista ``Júlio de Mesquita Filho'', Faculdade de Ciências, Campus Bauru.}
% ---


% ---
% Configurações de aparência do PDF final

% alterando o aspecto da cor azul
\definecolor{blue}{RGB}{41,5,195}

% informações do PDF
\makeatletter
\hypersetup{
     	%pagebackref=true,
		pdftitle={\@title},
		pdfauthor={\@author},
    	pdfsubject={\imprimirpreambulo},
	    pdfcreator={LaTeX with abnTeX2},
		pdfkeywords={abnt}{latex}{abntex}{abntex2}{trabalho acadêmico},
		colorlinks=true,       		% false: boxed links; true: colored links
    	linkcolor=black,          	% color of internal links
    	citecolor=black,        		% color of links to bibliography
    	filecolor=magenta,      		% color of file links
		urlcolor=black,
		bookmarksdepth=4
}
\makeatother
% ---

% ---
% Espaçamentos entre linhas e parágrafos
% ---

% O tamanho do parágrafo é dado por:
\setlength{\parindent}{1.3cm}

% Controle do espaçamento entre um parágrafo e outro:
\setlength{\parskip}{0.2cm}  % tente também \onelineskip

% ---
% compila o indice
% ---
\makeindex
% ---

% ----
% Início do documento
% ----
\begin{document}

% Seleciona o idioma do documento (conforme pacotes do babel)
%\selectlanguage{english}
\selectlanguage{brazil}

% Retira espaço extra obsoleto entre as frases.
\frenchspacing

% ----------------------------------------------------------
% ELEMENTOS PRÉ-TEXTUAIS
% ----------------------------------------------------------
% \pretextual

% ---
% Capa
% ---
\imprimircapa
% ---

% ---
% Folha de rosto
% (o * indica que haverá a ficha bibliográfica)
% ---
\imprimirfolhaderosto*
% ---

% ---
% Inserir a ficha bibliografica
% ---

% Isto é um exemplo de Ficha Catalográfica, ou ``Dados internacionais de
% catalogação-na-publicação''. Você pode utilizar este modelo como referência.
% Porém, provavelmente a biblioteca da sua universidade lhe fornecerá um PDF
% com a ficha catalográfica definitiva após a defesa do trabalho. Quando estiver
% com o documento, salve-o como PDF no diretório do seu projeto e substitua todo
% o conteúdo de implementação deste arquivo pelo comando abaixo:
%
% \begin{fichacatalografica}
%     \includepdf{fig_ficha_catalografica.pdf}
% \end{fichacatalografica}

\begin{fichacatalografica}
	\sffamily
	\vspace*{\fill}					% Posição vertical
	\begin{center}					% Minipage Centralizado
	\fbox{\begin{minipage}[c][8cm]{15.5cm}		% Largura
	\small
	\imprimirautor
	%Sobrenome, Nome do autor

	\hspace{0.5cm} \imprimirtitulo  / \imprimirautor. --
	\imprimirlocal, \imprimirdata-

	\hspace{0.5cm} \pageref{LastPage} p. : il. (algumas color.) ; 30 cm.\\

	\hspace{0.5cm} \imprimirorientadorRotulo~\imprimirorientador\\

	\hspace{0.5cm}
	\parbox[t]{\textwidth}{\imprimirtipotrabalho~--~\imprimirinstituicao,
	\imprimirdata.}\\

	\hspace{0.5cm}
		1. PenTest
		2. Cloud Computing
		3. Security
		I. \imprimirorientador.
		II. Universidade Estadual Paulista "Júlio de Mesquita Filho".
		III. Faculdade de Ciências.
		IV. Métodos de Teste de Penetração para Sistemas Web na Nuvem
	\end{minipage}}
	\end{center}
\end{fichacatalografica}
% ---

% ---
% Inserir folha de aprovação
% ---

% Isto é um exemplo de Folha de aprovação, elemento obrigatório da NBR
% 14724/2011 (seção 4.2.1.3). Você pode utilizar este modelo até a aprovação
% do trabalho. Após isso, substitua todo o conteúdo deste arquivo por uma
% imagem da página assinada pela banca com o comando abaixo:
%
% \includepdf{folhadeaprovacao_final.pdf}
%
\begin{folhadeaprovacao}

  \begin{center}
    {\ABNTEXchapterfont\large\imprimirautor}

    \vspace*{\fill}\vspace*{\fill}
    \begin{center}
      \ABNTEXchapterfont\bfseries\Large\imprimirtitulo
    \end{center}
    \vspace*{\fill}

    \hspace{.45\textwidth}
    \begin{minipage}{.5\textwidth}
        \imprimirpreambulo
    \end{minipage}%
    \vspace*{\fill}
   \end{center}

   \center Banca Examinadora

   \assinatura{\textbf{\imprimirorientador} \\ Orientador}
   \assinatura{\textbf{Prof. Dr. Simone das Graças Domingues Prado} }
   \assinatura{\textbf{Prof. Dr. } }

   \begin{center}
    \vspace*{0.5cm}
    \par
    {Bauru}
    {2015}
    \vspace*{1cm}
  \end{center}

\end{folhadeaprovacao}
% ---

% ---
% Dedicatória
% ---
\begin{dedicatoria}
   \vspace*{\fill}
   \centering
   \noindent
   \textit{Espaço destinado à dedicátoria do texto.} \vspace*{\fill}
\end{dedicatoria}
% ---

% ---
% Agradecimentos
% ---
\begin{agradecimentos}
Espaço destinado aos agradecimentos.
\end{agradecimentos}
% ---

% \textemdash
% Epígrafe
% ---
\begin{epigrafe}
    \vspace*{\fill}
	\begin{flushright}
		\textit{Espaço destinado à epígrafe.}
	\end{flushright}
\end{epigrafe}
% ---

% ---
% RESUMOS
% ---

% resumo em português
\setlength{\absparsep}{18pt} % ajusta o espaçamento dos parágrafos do resumo
\begin{resumo}

Espaço destinado à escrita do resumo.

\textbf{Palavras-chave:} Palavras-chave de seu resumo.
\end{resumo}

% resumo em inglês
\begin{resumo}[Abstract]
 \begin{otherlanguage*}{english}

Abstract area.

\textbf{Keywords:} Abstract keywords.

 \end{otherlanguage*}
\end{resumo}
% ---

% ---
% inserir lista de ilustrações
% ---
\pdfbookmark[0]{\listfigurename}{lof}
\listoffigures*
\cleardoublepage
% ---

% ---
% inserir lista de tabelas
% ---
\pdfbookmark[0]{\listtablename}{lot}
\listoftables*
\cleardoublepage
% ---

% ---
% inserir lista de abreviaturas e siglas
% ---
% ---

% ---
% inserir o sumario
% ---
\pdfbookmark[0]{\contentsname}{toc}
\tableofcontents*
\cleardoublepage
% ---



% ----------------------------------------------------------
% ELEMENTOS TEXTUAIS
% ----------------------------------------------------------
\pagestyle{simple}

% ----------------------------------------------------------
% Introdução (exemplo de capítulo sem numeração, mas presente no Sumário)
% ----------------------------------------------------------

\chapter{Introdução}
\label{c.introducao}

Com o grande crescimento que se deu na tecnologia nos últimos anos, surgiram novas maneiras para sanar quesitos de gastos e desempenho quanto às necessidades básicas. Os sistemas em nuvem ({\em Cloud Services}) surgiram de modo a terceirizar o hardware utilizado, sendo que o cliente paga apenas o que paga, deixando para este apenas se preocupar com o produto que será comercializado.

Como esses serviços se encarregam de manter a hospedagem de dados,além de manter o tráfico de usuários não congestionado, o cliente não se preocupa também com as ameaças que possa vir a ter. Nesse ponto, o sistema na nuvem terá que oferecer um sistema invunerável, para manter credibilidade.

Testes de Penetração, também conhecidos como {\em PenTest} ou Testes de Invasão, consistem em recolher informações sobre o alvo, identificar possíveis aberturas, tentativas de invasão e relatórios sobre o teste propriamente dito. O objetivo principal de um pentest é de determinar pontos fracos na segurança do sistema

\section{Problema}
\label{s.problema}

Justamente como a tecnologia está crescendo, novas ameaças aparecem diariamente. Ao oferecer um serviço que irá dispor de todos os dados de seus clientes, eles tem que possuir uma garantia de que não perderão seus dados.

Por esse motivo, é necessário manter em dia as possíveis vulnerabilidades, mantendo uma maior segurança para ambos os lados.


% \section{Abordagem Colaborativa}
% \label{s.abordagem}
%
% Segundo \citeauthoronline{unite10} (\citeyear{unite10}, Tradução nossa), projetos colaborativos permitem a criação conjunta e simultânea de conteúdo por muitos usuários finais e são, nesse sentido, provavelmente a mais democrática manifestação de conteúdo gerado pelo usuário.
%
% Ao desenvolver uma ferramenta que seja modular, é possível garantir uma maior coesão entre os componentes dessa ferramenta, pois cada módulo tem uma função definida e uma definição de formato de entrada e saída de dados.
%
% Por se trabalhar com código aberto o conteúdo disponível é altamente modificável, pois o código faz parte da implementação e da transmissão do conhecimento.
% Essa possibilidade de modificação faz com que o código seja continuamente refatorado e revisto, e dentro de padrões de desenvolvimento tende a melhorar sua qualidade final.
%
% Como diz \citeauthoronline{weber} (\citeyear{weber}, Tradução nossa) o código aberto não oblitera o lucro, o capitalismo, ou direitos intelectuais, e portanto não vai contra os modelos atuais de mercado, inclusive auxilia nos processos de todas as áreas sociais. O próprio formato do HTML\footnote{HyperText Markup Language (Linguagem de Marcação de Hipertexto).} e essa ampla distribuição de código aberto através da internet tem como efeito uma grande divulgação de linguagens e ferramentas open source. Ferramentas que antes precisavam de um conhecimento mais amplo em computação, agora podem ser manipuladas com uma chamada do script direto no navegador de quem está testando um pedaço de código.
%
% A utilização do formato modularizado garante também uma liberdade e uma intercomunicação entre trabalhos já desenvolvidos, como cálculos de transformadas e algoritmos de aprendizagem.%colocar mais libs usadas aqui

\chapter{Objetivos}
\label{c.objetivos}

\section{Objetivos Gerais}
\label{s.objetivosgerais}

Estudar metodologias de Teste de Penetração em \emph{Cloud Services}, analisando a possibilidade de criar uma ferramenta que automatize esta tarefa.

\section{Objetivos Específicos}
\label{s.objetivosespecificos}

\begin{alineas}
	\item Aprender metodologias de Teste de Penetração;
	\item Estudar sobre \emph{Cloud Service};
	\item Tentar prever qual será o futuro da \emph{Cloud Computing};
\end{alineas}

\chapter{Fundamentação Teórica}
\label{c.fundamentacaoteorica}

A utilização e implementação de bibliotecas externas, assim como a comunicação entre diferentes linguagens tem aproximado a interaçdizem que:

% \begin{quotation}
% ``Processos cognitivos envolvem trajetórias de informações (de transmissão e transformação), de modo que os padrões destas trajetórias de informação, se estáveis, refletem uma arquitetura cognitiva subjacente. Uma vez que a organização social - mais a estrutura adicionada ao contexto da atividade - determina em grande parte o modo como a informação flui através de um grupo, a organização social em si pode ser vista como uma forma de arquitetura cognitiva.''
% \end{quotation}

Dessa forma o desenvolvimento modular não se dá somente por convenção e facilidade de implementação, como também cumpre um papel de definição organizacional entre várias bibliotecas e estruturas de dados.

\section{Segurança}
\label{s.seguranca}

Um audio fingerprint é definido por  \citeauthoronline{canoetal05} (\citeyear{canoetal05}, tradução nossa) como uma assinatura única de uma música, contendo um sumário de suas caracterí

\section{Teste de Penetração}
\label{s.pentest}

\section{Teste de Penetração na Área Forense}
\label{s.pentestforense}

\section{Computação na Nuvem}
\label{s.cloudcomputing}

\section{Serviços na Nuvem}
\label{s.cloudservices}

\section{A importância de Testes de Penetração em Serviços na Nuvem}
\label{s.pentestincloud}

% \chapter{Metodologia}
\label{c.metodologia}

\section{Métodos e Etapas}
\label{s.metodoseetapas}

Para o desenvolvimento do projeto foi realizado o levantamento bibliográfico de tecnologias existentes e modelos organizacionais referentes à extração de \emph{Audio Fingerprint}, assim como a comparação específica de modelos de ondas sonoras de músicas e melodias.

Com a base teórica definida a etapa seguinte foi aplicar uma estrutura modular que correspondesse ao modelo teórico, para que todas as etapas da extração e reconhecimento do \emph{audio fingerprint} possam ser modificadas sem que alterem o funcionamento geral do processo. Essa estrutura adaptável foi aplicada com base no modelo teórico genérico de reconhecimento de áudio.

A terceira e última etapa foi o desenvolvimento de uma aplicação de reconhecimento específico de músicas, utilizando um cálculo de \emph{audio fingerprint} baseado em representações que levam em conta a repetição de refrões através da análise baseada em altura tonal e \emph{chroma} musical.

\section{Materiais Utilizados}
\label{s.materiaisutilizados}


\subsection{Servidor de desenvolvimento}
\label{s.v8}

Para desenvolvimento do projeto foi utilizada uma máquina externa com 512MB de RAM, Disco SSD com 20GB e Ubuntu 14.04 x64.

\subsection{Github}
\label{s.github}

O Github é ao mesmo tempo um servidor de armazenamento de código e uma rede social onde pode-se submeter modificações, fazer cópias e acompanhar modificações de códigos de outras pessoas. A rede foi essencial para o desenvolvimento deste projeto por armazenar vários sub-módulos e disponibilizar código-fonte para consulta.

\subsection{V8 JavaScript Engine}
\label{s.v8}

V8 é um projeto open-source da Google para interpretação de Javascript, escrito em C++ e utilizado no navegador \emph{Google Chrome}. Ele implementa o ECMAScript e pode ser inserido em aplicações C++ nativas.

\subsection{Node.js}
\label{s.nodejs}

O Node.js é um interpretador \emph{JavaScript} multiplataforma que utiliza o \emph{V8 JavaScript Engine}, foi criado em 2009 por Ryan Dahl e atualmente é utilizado para aplicações variadas. Seu funcionamento \emph{single threaded} simplifica a construção da aplicação, foi escolhido para o desenvolvimento deste projeto por facilitar o desenvolvimento modular e comunicação entre os módulos.

\subsection{npm}
\label{s.npm}

O npm é uma sigla para \emph{node package manager} mas é um gerenciador de pacotes geral que implementa não somente os pacotes de node como também de projetos variados em javascript e outras linguagens. O npm permite a instalação e gerenciamento de pacotes e bibliotecas de maneira facilitada, coordenando as versões utilizadas de acordo com a necessidade do projeto.

\subsection{FFmpeg}
\label{s.ffmpeg}

O FFmpeg é um programa externo multiplataforma utilizado através do node. Ele grava, converte e cria \emph{stream} de vários formatos de áudio e vídeo.

% \input{desenvolvimento.tex}
% \input{conclusao.tex}


% ----------------------------------------------------------
% ELEMENTOS PÓS-TEXTUAIS
% ----------------------------------------------------------
\postextual
% ----------------------------------------------------------

% ----------------------------------------------------------
% Referências bibliográficas
% ----------------------------------------------------------
\pagestyle{empty}
\bibliography{references} % o arquivo de bibliografia deve ser importando nessa linha sem o .bib

% ----------------------------------------------------------
% Glossário
% ----------------------------------------------------------
%
% Consulte o manual da classe abntex2 para orientações sobre o glossário.
%
%\glossary

%---------------------------------------------------------------------
% INDICE REMISSIVO
%---------------------------------------------------------------------
\phantompart
\printindex
%---------------------------------------------------------------------

\end{document}
