\chapter{Fundamentação Teórica}
\label{c.fundamentacaoteorica}

Testes de Penetração, ou {\em PenTests}, são práticas ao se testar um sistema, rede ou aplicações Web, procurando possíveis vulnerabilidades que poderiam vir a ser utilizadas para algum ataque.

A metodologia pode variar para cada sistema, mas basicamente se resume a:

\begin{alineas}
  \item Reunir informações sobre o alvo do teste (reconhecimento);
  \item Identificar possíveis falhas de segurança;
  \item Tentativa de invasão, seja virtualmente ou pessoalmente;
  \item E por fim, gerar um relatório das descobertas.
\end{alineas}

% \begin{quotation}
% ``Processos cognitivos envolvem trajetórias de informações (de transmissão e transformação), de modo que os padrões destas trajetórias de informação, se estáveis, refletem uma arquitetura cognitiva subjacente. Uma vez que a organização social - mais a estrutura adicionada ao contexto da atividade - determina em grande parte o modo como a informação flui através de um grupo, a organização social em si pode ser vista como uma forma de arquitetura cognitiva.''
% \end{quotation}

\section{Segurança}
\label{s.seguranca}

\section{Teste de Penetração}
\label{s.pentest}

\section{Computação na Nuvem}
\label{s.cloudcomputing}

\section{Serviços na Nuvem}
\label{s.cloudservices}

\section{Teste de Penetração como Área Forense}
\label{s.pentestforense}

\section{A importância de Testes de Penetração em Serviços na Nuvem}
\label{s.pentestincloud}
