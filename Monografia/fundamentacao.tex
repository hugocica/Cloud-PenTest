\chapter{Fundamentação Teórica}
\label{c.fundamentacaoteorica}

Testes de Penetração, ou {\em PenTests}, são práticas realizadas para determinar fraquezas na segurança de algum sistema, este podendo ser um Sistema de Computador, uma Rede de Computadores ou uma Aplicação Web, por exemplo. Além de determinar fraquezas, pode ser usado em empresas para verificar o quão apto estão seus funcionários para terem consciência das fraquezas e como irão responder diante de um ataque.

Considera-se um ataque toda e qualquer invasão que um Sistema Computacional pode sofrer, sem aviso prévio. Este ataque geralmente é realizado por alguém com um certo conhecimento técnico na área de programação e segurança. Esta pessoa é comumente chamada de {\em Hacker}.

Um {\em Hacker} geralmente é considerado uma pessoa cujo intuito é causa problemas como, por exemplo, inserindo vírus, roubando números de cartão de crédito. Porém, segundo \citeauthoronline{hackervscracker}, o nome para pessoas que buscam se aproveitar de falhas em sistemas para ganho pessoal é dado de {\em Crackers}.

De acordo com \citeauthoronline{hackerdictionary} (\citeyear{hackerdictionary}), {\em Hacker} é definido como um programador engenhoso, um bom {\em hackeamento} é uma solução engenhosa para um problema de programação e {\em hackear} é o ato de solucionar este problema. \citeauthoronline{hackerdictionary} ainda cita cinco possíveis características que qualificam um {\em hacker}:

\begin{alineas}
  \item Uma pessoa a qual aprecia aprender detalhes de uma linguagem de programação ou de um sistema;
  \item Uma pessoa a qual aprecia programar ao invés de apenas teorizar;
  \item Uma pessoa capaz de apreciar o {\em hackeamento} de outra pessoa;
  \item Uma pessoa que aprende rapidamente uma linguagem de programação;
  \item Uma pessoa que é perito em determinada linguagem de programação ou sistema computacional.
\end{alineas}

Portanto, um {\em PenTester}, pessoa que responsável por realizar tarefas de Testes de Penetração, pode ser considerado como {\em Hacker}.

Como Testes de Penetração tentam explorar fraquezas em algum Sistema Computacional, fornecendo relatórios sobre essas possíveis fraquezas com finalidade de solucioná-las, este pode ser incorporado como uma área de Segurança de Computadores.

% \begin{quotation}
% ``Processos cognitivos envolvem trajetórias de informações (de transmissão e transformação), de modo que os padrões destas trajetórias de informação, se estáveis, refletem uma arquitetura cognitiva subjacente. Uma vez que a organização social - mais a estrutura adicionada ao contexto da atividade - determina em grande parte o modo como a informação flui através de um grupo, a organização social em si pode ser vista como uma forma de arquitetura cognitiva.''
% \end{quotation}

\section{Segurança}
\label{s.seguranca}

Desde sempre, o valor da informação foi essencial, e nos tempos atuais não é diferente. Com o aumento da tecnologia, existe uma grande preferência de manter informações virtualizadas, seja pela grande capacidade de armazenamento que os sistemas computacionais possuem, ou pelo fácil acesso que se dá devido à internet, por exemplo. Porém, manter informações centralizadas em um só lugar pode se tornar um problema, caso não mantenha a segurança de seus dispositivos atualizada.

Para proteger dados e informações tendo em mente as diversas arquiteturas de rede que existem, {\em Web Services}, aplicações, diferentes plataformas de servidores, está mais difícil do que nunca. Por os computadores e a internet de fato estar presente no nosso dia-a-dia nas últimas décadas, invasões não são mais realizadas por crianças curiosas se aventurando no mundo dos códigos.

Apesar de existirem métodos os quais previnem o roubo de dados ou invasão de sistemas, como é o caso de Anti-vírus, por exemplo, o melhor jeito de descobrir se um sistema está realmente seguro contra invasões é tentando invadir o mesmo, pois apenas assim será possível detectar problemas reais que {\em hackers} mal intesionados podem vir a causar.

Dessa forma, é possível ver Testes de Penetração como um ramo da Segurança de Computadores, tendo em vista de que eles são realizados visando melhor as atuais falhas que pode vir a se descobrir com o avanço de novos métodos.

\section{Teste de Penetração}
\label{s.pentest}

Como já explicamos superficialmente no início deste capítulo, Testes de Penetração, ou também chamados Testes de Invasão ou {\em PenTest}, são testes realizados de modo a simular um ataque mal intencionado à uma rede, aplicação ou sistema computacional.

O objetivo de se realizar Testes de Invasão é justamente para aumentar a segurança dos sistemas testados, prevenindo assim que dados e informações sejam roubadas ou manipuladas por um {\em Cracker}. {\em Crackers}, como já foi dito, são {\em Hackers} que violam o Código de Ética dos {\em Hackers}, segundo \citeauthoronline{hackerdictionary}. A linha que separa um {\em PenTester} de um {\em Cracker} é exatamente pelo fato de um {\em PenTester} ter sido contratado para realizar o ato de invadir o sistema, tendo uma permissão prévia concedida pelo dono do sistema.

Testes de Invasão muitas vezes são confundidos com Avaliação de Vulnerabilidade. Porém isso é um erro comum, visto que o foco o qual se dá ao realizar um Teste de Penetração é na tentativa de ganhar um acesso ao sistema, a ponto de não existirem restrições a arquivos e dados sigilosos, enquanto que Avaliação de Vulnerabilidade tem ênfase em identificar pontos vulneráveis na segurança, não focando em tentar invadí-los.

Um {\em PenTester} tem como responsabilidade tentar invadir o sistema desejado de modo que irá usar os meios e métodos que um {\em Cracker} também usaria, para garantir que o sistema está seguro. Durante o processo de invasão, o {\em PenTester} estará fazendo um relatório detalhado, passando por todos os caminhos que ele tenha tomado, para que no final esteja especificado quais são as falhas do sistema, como foram descobertas e assim arrumá-las para que não haja potencial de invasão. Contudo, precisa-se ter em mente de que uma falha corrigida ontem, pode resultar em uma falha imprevista hoje. Por conta disso, é importante manter constante as realizações de testes de penetração, garantindo assim um sistema seguro.

% \section{Computação na Nuvem}
% \label{s.cloudcomputing}

% \section{Serviços na Nuvem}
% \label{s.cloudservices}

% \section{Teste de Penetração como Área Forense}
% \label{s.pentestforense}

% \section{Testes de Penetração em Sistemas na Nuvem}
% \label{s.pentestincloud}

% \section{A importância de Testes de Penetração em Serviços na Nuvem}
% \section{A importância de Testes de Penetração}
% \label{s.pentestimportant}
