\chapter{Fundamentação Teórica}
\label{c.fundamentacaoteorica}

Testes de Penetração, ou {\em PenTests}, são práticas realizadas para determinar fraquezas na segurança de algum sistema, este podendo ser um Sistema de Computador, uma Rede de Computadores ou uma Aplicação Web, por exemplo. Além de determinar fraquezas, pode ser usado em empresas para verificar o quão apto estão seus funcionários para terem consciência das fraquezas e como irão responder diante de um ataque.

Considera-se um ataque toda e qualquer invasão que um Sistema Computacional pode sofrer, sem aviso prévio. Este ataque geralmente é realizado por alguém com um certo conhecimento técnico na área de programação e segurança. Esta pessoa é comumente chamada de {\em Hacker}.

Um {\em Hacker} geralmente é considerado uma pessoa cujo intuito é causa problemas como, por exemplo, inserindo vírus, roubando números de cartão de crédito. Porém, segundo \citeauthoronline{hackervscracker}, o nome para pessoas que buscam se aproveitar de falhas em sistemas para ganho pessoal é dado de {\em Crackers}.

De acordo com \citeauthoronline{hackerdictionary} (\citeyear{hackerdictionary}), {\em Hacker} é definido como um programador engenhoso, um bom {\em hackeamento} é uma solução engenhosa para um problema de programação e {\em hackear} é o ato de solucionar este problema. \citeauthoronline{hackerdictionary} ainda cita cinco possíveis características que qualificam um {\em hacker}:

\begin{alineas}
  \item Uma pessoa a qual aprecia aprender detalhes de uma linguagem de programação ou de um sistema;
  \item Uma pessoa a qual aprecia programar ao invés de apenas teorizar;
  \item Uma pessoa capaz de apreciar o {\em hackeamento} de outra pessoa;
  \item Uma pessoa que aprende rapidamente uma linguagem de programação;
  \item Uma pessoa que é perito em determinada linguagem de programação ou sistema computacional.
\end{alineas}

Portanto, um {\em PenTester}, pessoa que responsável por realizar tarefas de Testes de Penetração, pode ser considerado como {\em Hacker}.

Como Testes de Penetração tentam explorar fraquezas em algum Sistema Computacional, fornecendo relatórios sobre essas possíveis fraquezas com finalidade de solucioná-las, este pode ser incorporado como uma área de Segurança de Computadores.

% \begin{quotation}
% ``Processos cognitivos envolvem trajetórias de informações (de transmissão e transformação), de modo que os padrões destas trajetórias de informação, se estáveis, refletem uma arquitetura cognitiva subjacente. Uma vez que a organização social - mais a estrutura adicionada ao contexto da atividade - determina em grande parte o modo como a informação flui através de um grupo, a organização social em si pode ser vista como uma forma de arquitetura cognitiva.''
% \end{quotation}

\section{Segurança de Computadores}
\label{s.seguranca}

Desde sempre

Embora o objetivo de

% \section{Teste de Penetração}
% \label{s.pentest}

% \section{Computação na Nuvem}
% \label{s.cloudcomputing}

% \section{Serviços na Nuvem}
% \label{s.cloudservices}

% \section{Teste de Penetração como Área Forense}
% \label{s.pentestforense}

% \section{Testes de Penetração em Sistemas na Nuvem}
% \label{s.pentestincloud}

% \section{A importância de Testes de Penetração em Serviços na Nuvem}
% \section{A importância de Testes de Penetração}
% \label{s.pentestimportant}
